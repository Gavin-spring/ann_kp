% !TEX root = ../Dissertation.tex
\section{Motivation}

The 0/1 Knapsack Problem (KP) is a canonical problem in the field of combinatorial optimization and computer science, formally classified as NP-complete \cite{cappartCombinatorialOptimizationReasoning2022}. Its importance is twofold. Theoretically, as many NP-complete problems are reducible to one another, advancements in solving the KP can provide insights applicable to a wide range of other computationally hard problems. Practically, the KP serves as a mathematical abstraction for numerous real-world decision-making scenarios, including portfolio optimization, resource allocation, and logistics planning.

Despite its significance, the development of scalable and generalizable solvers remains a major challenge. Traditional exact algorithms, while guaranteeing optimality, are not viable for large-scale instances that are common in industrial applications. This has motivated a growing interest in machine learning-based approaches \cite{wangSolvingCombinatorialOptimization2024}. However, a review of recent literature reveals a critical research gap: the majority of existing neural network-based solvers are designed and trained for specific, fixed problem sizes \cite{belloNeuralCombinatorialOptimization2017}. These models often fail to generalize their learned policies to instances with a different number of items than they were trained on, severely limiting their practical utility.

Furthermore, within the sub-field of deep reinforcement learning for combinatorial optimization, the fundamental 0/1 Knapsack Problem has received surprisingly little attention compared to other problems like the Traveling Salesperson Problem. The lack of a truly end-to-end framework that can be trained on smaller instances and effectively generalize to larger, unseen ones represents a significant and compelling opportunity for research. This dissertation is motivated by the goal of addressing this gap by developing a novel RL framework that learns a robust and scalable policy for the 0/1 Knapsack Problem.